% !TeX root = ../main.tex

\chapter{关于论文写作的规定}

\section{中山大学研究生学位论文字数要求}
\begin{itemize}
    \item 硕士论文: 正文一般为$1 \sim 3$万字
    \item 博士论文: 正文一般不超过$15$万字
\end{itemize}



<<<<<<< HEAD
\section{正文的写法}

本部分是论文作者的研究内容,不能将他人研究成果不加区分地掺和进来。
已经在引言的文献综述部分讲过的内容,这里不需要再重复。
各章之间要存在有机联系,符合逻辑顺序。



\section{结论的写法}

结论是对论文主要研究结果、论点的提炼与概括,应精炼、准确、完整,使读者看后能全面了解论文的意义、目的和工作内容。
结论是最终的、总体的结论,不是正文各章小结的简单重复。
结论应包括论文的核心观点,主要阐述作者的创造性工作及所取得的研究成果在本领域中的地位、作用和意义,交代研究工作的局限,提出未来工作的意见或建议。
同时,要严格区分自己取得的成果与指导教师及他人的学术成果。

在评价自己的研究工作成果时,要实事求是,除非有足够的证据表明自己的研究是“首次”、“领先”、“填补空白”的,否则应避免使用这些或类似词语。
\section{占页}
占页
\section{占页}
占页
\section{占页}
占页

\cleardoublepage
=======
>>>>>>> d70781f2ad2928e37c5c6c232f723bdce549f502
