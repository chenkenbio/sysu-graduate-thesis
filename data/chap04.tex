% !TeX root = ../main.tex

\chapter{数学符号和公式}\label{equations}

\section{数学符号}

论文中使用的符号应符合国家标准《国际单位制及其应用》(GB 3100—1993)、《有关量、单位和符号的一般原则》(GB/T 3101—1993) 的规定。

要特别注意以下几点问题:

\begin{enumerate}
\item 数学公式结尾也不可缺少标点符号, 参见\eqref{eq:example}, \eqref{eq:align}, \eqref{eq:equ_aligned}.
    \item 向量、矩阵和张量要求粗斜体,应该使用 \pkg{unicode-math} 的 \cs{symbf} 命令,
          如 \verb|\symbf{A}|、\verb|\symbf{\alpha}|。
    \item 数学常数和特殊函数要求用正体,应使用 \cs{symup} 命令,
    如 $\symup{\pi} = 3.14\dots$; $\symup{e} = 2.718\dots$。
    \item 微分号和积分号使用使用正体,比如:$\int f(x) \dif x$。
    \item 公式中的括号应写作 $\displaystyle \left(\frac{a}{b + c}\right)$, $\displaystyle \left[\frac{a}{b + c}\right]$, $\displaystyle \left\{\frac{a}{b + c}\right\}$, $\displaystyle \left<\frac{a}{b + c}\right>$, 而非 $\displaystyle (\frac{a}{b + c})$, $\displaystyle [\frac{a}{b + c}]$, $\displaystyle \{\frac{a}{b + c}\}$, $\displaystyle <\frac{a}{b + c}>$.\\
    注意比较两种形式括号大小的区别. 在\LaTeX{}可用 \verb|\left(| $\cdot$ \verb|\right)|, \verb|\left[| $\cdot$ \verb|\right]|, \verb|\left\{| $\cdot$ \verb|\right\}|, \verb|\left<| $\cdot$ \verb|\right>| 来实现.
    \item 关于量和单位推荐使用\href{http://mirrors.ctan.org/macros/latex/contrib/siunitx/siunitx.pdf}{\pkg{siunitx}}
宏包,
        可以方便地处理希腊字母以及数字与单位之间的空白,
        比如:
        \SI{6.4e6}{m},
        \SI{9}{\micro\meter},
        \si{kg.m.s^{-1}},
        \SIrange{10}{20}{\degreeCelsius}。
\end{enumerate}

\section{数学公式}

数学公式可以使用 \env{equation} 和 \env{equation*} 环境。
注意数学公式的引用应前后带括号,建议使用 \cs{eqref} 命令,比如式 \eqref{eq:example},
\begin{equation}
  \frac{1}{2 \symup{\pi} \symup{i}} \int_\gamma f = \sum_{k=1}^m n\left(\gamma; a_k\right) \mathscr{R}\left(f; a_k\right).
  \label{eq:example}
\end{equation}

多行公式尽可能在 ``='' 处对齐,可根据编号形式的需要使用 \env{align} 或 \env{equation}+\env{aligned} 环境, 请参考如下示例:

\env{align}模式: 

\begin{minipage}{.48\textwidth}
    \begin{verbatim}
\begin{align}
  a & = b + c + d + e, \\
    & = f + g.
\end{align}
    \end{verbatim}
\end{minipage}
\begin{minipage}{.48\textwidth}
  \begin{align}\label{eq:align}
    a & = b + c + d + e, \\
      & = f + g.
  \end{align}
\end{minipage}

\env{equation} + \env{aligned} 模式:

\begin{minipage}{.48\textwidth}
    \begin{verbatim}
\begin{equation}
    \begin{aligned}
      a & = b + c + d + e, \\
        & = f + g.
    \end{aligned}
\end{equation}
    \end{verbatim}
\end{minipage}
\begin{minipage}{.48\textwidth}
    \begin{equation}\label{eq:equ_aligned}
        \begin{aligned}
          a & = b + c + d + e, \\
            & = f + g.
        \end{aligned}
    \end{equation}
\end{minipage}

\section{数学定理}

定理环境的格式可以使用 \pkg{amsthm} 或者 \pkg{ntheorem} 宏包配置。
用户在导言区载入这两者之一后,模板会自动配置 \env{thoerem}、\env{proof} 等环境。

\begin{theorem}[Lindeberg--Lévy 中心极限定理]
  设随机变量 $X_1, X_2, \dots, X_n$ 独立同分布, 且具有期望 $\mu$ 和有限的方差 $\sigma^2 \ne 0$,
  记 $\bar{X}_n = \frac{1}{n} \sum_{i+1}^n X_i$,则
  \begin{equation}
    \lim_{n \to \infty} P \left(\frac{\sqrt{n} \left( \bar{X}_n - \mu \right)}{\sigma} \le z \right) = \Phi(z),
  \end{equation}
  其中 $\Phi(z)$ 是标准正态分布的分布函数。
\end{theorem}
\begin{proof}
  Trivial.
\end{proof}

同时模板还提供了 \env{assumption}、\env{definition}、\env{proposition}、
\env{lemma}、\env{theorem}、\env{axiom}、\env{corollary}、\env{exercise}、
\env{example}、\env{remar}、\env{problem}、\env{conjecture} 这些相关的环境。

\cleardoublepage
